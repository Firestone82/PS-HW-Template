\documentclass[a4paper,12pt]{article}
\usepackage[utf8]{inputenc}
\usepackage[czech]{babel}
\usepackage{amsmath}
\usepackage{geometry}
\usepackage{xcolor}
\usepackage{graphicx}
\usepackage{fancyhdr}
\usepackage{xcolor}
\usepackage{soul}

\geometry{top=2.5cm, bottom=2.5cm, left=2.5cm, right=2cm}

\renewcommand{\headrulewidth}{0pt}
\renewcommand{\footrulewidth}{0pt}
\pagestyle{fancy}
\fancyhead[L]{\colorbox{yellow}{\parbox{\dimexpr\textwidth-2\fboxsep\relax}{Jméno:}}}
\fancyhead[C]{\colorbox{yellow}{\parbox{\dimexpr\textwidth-2\fboxsep\relax}{}}}
\fancyhead[R]{\colorbox{yellow}{}{Číslo zadání: X}}

\fancyfoot[C]{\thepage} % C = center (střed)


\begin{document}
\thispagestyle{empty}
% Title and header section
\begin{center}
    \includegraphics[width=0.8\textwidth]{logo.png} \\[4em]
    \textbf{PRAVDĚPODOBNOST A STATISTIKA} \\
    \vspace{1em}
    \textbf{Domácí úkoly 1S \textendash\ 3S} \\
    \textbf{\colorbox{yellow}{Zadání X}} \\
\end{center}


\vspace{2em}
\noindent
\textbf{Jméno studentky/studenta:} \\
\textbf{Osobní číslo:} \\
\textbf{Jméno cvičící/cvičícího:} \\


\begin{center}
\vspace{10em}
\resizebox{400pt}{50pt}{
    \begin{tabular}{|c|c|c|}
        \hline
        & \textbf{Datum odevzdání} & \textbf{Hodnocení} \\
        \hline
        Domácí úkol 1: & & \\
        \hline
        Domácí úkol 2: & & \\
        \hline
        Domácí úkol 3: & & \\
        \hline
        Celkem: & --------------------- & \\
        \hline
    \end{tabular}
}
\end{center}
% Footer section
\vspace{5em}
\begin{center}
\textbf{Ostrava, AR 2024/2025}
\end{center}
\newpage
\setcounter{page}{1}
\textbf{Popis datového souboru}
\\V datovém souboru jsou zaznamenány výkonnostní skóre (FPS) čtyř populárních grafických karet:
Nvidia RTX 2080 Ti, Nvidia RTX 3070 Ti, AMD Radeon RX 6800 XT a AMD Radeon RX 7700 XT. Tyto karty
byly testovány ve hře "Cyberpunk 2077" ve dvou různých verzích: původní release a po aplikaci 1.5
patche. Vaším úkolem je analyzovat, jak patch 1.5 ovlivnil výkonnostní skóre těchto karet ve hře. Pro
každý unikátní testovaný systém (test) viz (id) byly testovány obě verze hry\\\\
V souboru ukol\_X.xlsx jsou pro každý test uvedeny následující údaje:

\begin{itemize}
  \item \textbf{id} ... identifikátor testovaného systému (každý systém je unikátní PC systém)
  \item \textbf{typ karty} ... Nvidia RTX 2080 Ti, Nvidia RTX 3070 Ti, AMD Radeon RX 6800 XT, AMD Radeon RX
7700 XT
  \item  \textbf{testovaná verze} ... „release“ a „patched“
  \item \textbf{FPS} ... naměřené výkonnostní skóre (FPS) pro danou grafickou kartu s danou verzi hry.
\end{itemize}

\textbf{Obecné pokyny:}

\begin{itemize}
    \item Domácí úkoly odevzdávejte vždy v termínu, který určil váš cvičící.
    \item Portfolio domácích úkolů budete odevzdávat postupně. Tj. nejdříve odevzdáte titulní stránku
s úkolem 1, k okomentovanému úkolu 1 připojíte úkol 2 atd
    \item Domácí úkoly zpracujte dle obecně známých typografických pravidel. 
    \item Všechny tabulky i obrázky musí být opatřeny titulkem, který obsahuje i očíslování objektu.  
    \item Do domácích úkolů nevkládejte tabulky a obrázky, na něž se v doprovodném textu nebudete odkazovat.
    \item Bude-li to potřeba, citujte zdroje dle mezinárodně platné citační normy ČSN ISO 690.
\end{itemize}
\newpage
\textbf{Úkol 1}\\
Pomocí nástrojů explorační analýzy zkoumejte nárůst výkonnostních skóre (FPS) po aplikaci 1.5 patche
(tj. rozdíl výkonnostních skóre pro verzi „patched“ a verzi „release“) ve hře "Cyberpunk 2077" pro
grafické karty Nvidia RTX 3070 Ti a AMD Radeon RX 7700 XT. Data vhodně graficky prezentujte
(krabicový graf, histogram, q-q graf) a doplňte následující tabulku a text.\\\\
Výsledky popisné statistiky lze vidět v Tab. 1 a na ... (doplňte).

\begin{table}[h!]
    %\centering
    \renewcommand{\arraystretch}{1.3}
    \begin{tabular}{|p{3.5cm}|p{3cm}|p{3cm}|p{3cm}|p{3cm}|}
        \hline
        & \multicolumn{2}{p{6cm}|}{\centering \textbf{Původní data}} & \multicolumn{2}{p{6cm}|}{\centering \textbf{Data po odstranění odlehlých pozorování}} \\
        \hline
        & \centering \textbf{Nvidia RTX 3070 Ti} & \textbf{AMD Radeon RX 7700 XT} & \centering \textbf{Nvidia RTX 3070 Ti} & \textbf{AMD Radeon RX 7700 XT}\\
        \hline
        rozsah souboru & & & & \\
        \hline
        minimum & & & & \\
        \hline
        dolní kvartil & & & & \\
        \hline
        medián & & & & \\
        \hline
        průměr & & & & \\
        \hline
        horní kvartil & & & & \\
        \hline
        maximum & & & & \\
        \hline
        směrodat. odchylka & & & & \\
        \hline
        variační koeficient (\%) & & & & \\
        \hline
        šikmost & & & & \\
        \hline
        špičatost & & & & \\
        \hline
        \multicolumn{5}{|p{16cm}|}{\textbf{Identifikace odlehlých pozorování (vnitřní hradby)}} \\
        \hline
        dolní mez &  & &  & \\
        \hline
        horní mez &  & &  & \\
        \hline
    \end{tabular}
    \caption{Nárůst výkonnostních skóre (FPS) po aplikaci 1.5 patche ve hře "Cyberpunk 2077" pro grafické karty Nvidia RTX 3070 Ti a AMD Radeon RX 7700 XT (souhrnné statistiky}
\end{table}
\newpage
\textbf{Grafická prezentace (krabicový graf, histogram, q-q graf)}
\newpage
\textbf{Analýza nárůstu výkonnostních skóre (FPS) po aplikaci 1.5 patche ve hře "Cyberpunk 2077" pro grafickou kartu Nvidia RTX 3070 Ti}\\

Během testu byl zjišťován nárůst FPS pro grafickou kartu Nvidia RTX 3070 Ti ve hře "Cyberpunk 2077" mezi původním release a verzí s 1.5 patchem pro ... testovacích systémů. Zjištěný nárůst FPS se pohyboval v rozmezí ... FPS až ... FPS. \ul{Nárůst FPS v testu č. ...  byl
na základě metody vnitřních hradeb identifikován jako odlehlé pozorování a nebude zahrnut do dalšího zpracování.\footnote{V případě potřeby (existence vícero odlehlých pozorování) větu vhodným způsobem upravte.} Možné příčiny vzniku odlehlých pozorování jsou: ... / Žádný z nárůstů FPS nebyl identifikován jako odlehlé pozorování.} Dále uvedené výsledky tedy pocházejí z analýzy nárůstů FPS zjištěných u ... testovacích systémů. Průměrný nárůst FPS byl ... FPS, směrodatná odchylka pak ... FPS. U poloviny testovacích cyklů nárůst FPS nepřekročil ... FPS. V polovině případů se nárůst FPS pohyboval v rozmezí ... FPS až ... FPS. \ul{Vzhledem k hodnotě variačního koeficientu (...\%) lze / nelze analyzovaný soubor považovat za homogenní./ Vzhledem k povaze měřené veličiny není variační koeficient vhodnou mírou pro posouzení variability souboru.}\\

Výsledky pro grafickou kartu AMD Radeon RX 7700 XT lze komentovat obdobně.\\

\textbf{Ověření normality nárůstu výkonnostních skóre (FPS) po aplikaci 1.5 patche ve hře "Cyberpunk 2077" pro grafickou kartu Nvidia RTX 3070 Ti}\\\\
Na základě grafického zobrazení (viz ...) a výběrové šikmosti a špičatosti (výběrová šikmost
i špičatost \ul{leží \slash neleží} v intervalu (-2, 2) \ul{lze \slash nelze} předpokládat, že pozorovaný nárůst FPS má normální rozdělení. Dle \ul{pravidla 3$\sigma$ \slash Čebyševovy nerovnosti} lze tedy očekávat, že \ul{přibližně u 95 \% \slash
alespoň 75 \%} naměřených nárůstů ve výkonu bude ... FPS až ... FPS.\\\\

\textbf{Ověření normality nárůstu výkonnostních skóre (FPS) po aplikaci 1.5 patche ve hře "Cyberpunk 2077" pro grafickou kartu AMD Radeon RX 7700 XT}
Na základě grafického zobrazení (viz ...) a výběrové šikmosti a špičatosti (výběrová šikmost
i špičatost \ul{leží \slash neleží} v intervalu (-2, 2) \ul{lze \slash nelze} předpokládat, že pozorovaný nárůst FPS má normální rozdělení. Dle \ul{pravidla 3$\sigma$ \slash Čebyševovy nerovnosti} lze tedy očekávat, že \ul{přibližně u 95 \% \slash
alespoň 75 \%} naměřených nárůstů ve výkonu bude ... FPS až ... FPS.
\newpage
\textbf{Úkol 2}\\\\
Porovnejte nárůsty ve výkonnostních skórech (FPS) pro verzi hry "Cyberpunk 2077" po aplikaci 1.5
patche (dále jen \textbf{„nárůst FPS“}) pro vybrané grafické karty. Nezapomeňte, že použité metody mohou
vyžadovat splnění určitých předpokladů. Pokud tomu tak bude, okomentujte splnění/nesplnění těchto
předpokladů jak na základě explorační analýzy (např. s odkazem na histogram apod.), tak exaktně
pomocí metod statistické indukce.\\


\begin{enumerate}
    \begin{enumerate}
        \item Graficky prezentujte srovnání nárůstu FPS pro grafické karty Nvidia RTX 3070 Ti a AMD Radeon RX
7700 XT (vícenásobný krabicový graf, histogramy, q-q grafy). Srovnání okomentujte (včetně
informace o případné manipulaci s datovým souborem). \textbf{Poznámka:} Byla-li grafická prezentace FPS
v úkolu 1 bez připomínek, stačí do komentáře vložit odkaz na grafické výstupy z úkolu 1.


\newpage
    \item Na hladině významnosti 5 \% rozhodněte, zda jsou střední hodnoty nárůstů FPS (popř. mediány
nárustů FPS) pro grafické karty Nvidia RTX 3070 Ti a AMD Radeon RX 7700 XT statisticky významné.
K řešení využijte bodové a intervalové odhady i čistý test významnosti. Výsledky okomentujte.
\newpage
    \item Pro grafické karty Nvidia RTX 3070 Ti a AMD Radeon RX 7700 XT rozhodněte (na hladině
významnosti 5 \%), zda se jejich střední hodnoty (popř. mediány) nárůstu FPS po aplikaci 1.5 patche
statisticky významně liší. K řešení využijte bodový a intervalový odhad i čistý test významnosti.
Výsledky okomentujte.
    \end{enumerate}
\end{enumerate}
\newpage
\textbf{Úkol 3}\\
Na hladině významnosti 5 \% rozhodněte, zda střední hodnoty (popř. mediány) nárůstů FPS po aplikaci
1.5 patche statisticky významně závisí na typu grafické karty. Posouzení proveďte nejprve na základě
explorační analýzy a následně pomocí vhodného statistického testu, včetně ověření potřebných
předpokladů. V případě, že se nárůst FPS pro různé grafické karty statisticky významně liší, určete
pořadí karet dle středního nárůstu FPS (popř. mediánu nárůstu FPS). \textbf{Poznámka:} Srovnání proveďte pro
všechny čtyři typy grafických karet.
\begin{enumerate}
    \begin{enumerate}
        \item Daný problém vhodným způsobem graficky prezentujte (vícenásobný krabicový graf, histogramy,
q-q grafy). Srovnání okomentujte (včetně informace o případné manipulaci s datovým souborem).
\newpage
\item Ověřte normalitu a symetrii nárůstů FPS u všech čtyř grafických karet (empiricky i exaktně).
\newpage
\item Ověřte homoskedasticitu (shodu rozptylů) nárůstů FPS mezi jednotlivými kartami (empiricky
i exaktně)
\newpage
\item Určete bodové a 95\% intervalové odhady střední hodnoty (popř. mediánu) nárůstů FPS pro
všechny srovnávané karty. Volbu charakteristik proveďte tak, aby byly v souladu se statistickým
testem, který plánujete provést v bodě e). (Nezapomeňte na ověření předpokladů pro použití
příslušných intervalových odhadů.)
\newpage
\item Čistým testem významnosti ověřte, zda je pozorovaný rozdíl středních hodnot (popř. mediánů)
nárůstů FPS statisticky významný na hladině významnosti 5 \%. Pokud ano, zjistěte, zda lze některé
skupiny karet označit (z hlediska nárůstů FPS) za homogenní, tj. určete pořadí karet dle středních
hodnot (popř. mediánů) nárůstů FPS. (Nezapomeňte na ověření předpokladů pro použití
zvoleného testu.)
    \end{enumerate}
\end{enumerate}
\newpage

\subsection*{\textbf{Jak identifikovat, zda jsou v datech odlehlá pozorování?}}
\ul{Empirické posouzení:}
\begin{itemize}
    \item použití vnitřních (vnějších) hradeb
    \item vizuální posouzení krabicového grafu.
\end{itemize}

Jak naložit s odlehlými hodnotami by měl definovat hlavně zadavatel analýzy (expert na danou
problematiku).

\subsection*{Jak	ověřit	normalitu	dat?}
\ul{Empirické posouzení:}
\begin{itemize}
    \item vizuální posouzení histogramu,
    \item vizuální posouzení grafu odhadu hustoty pravděpodobnosti,
    \item Q-Q graf,
    \item posouzení výběrové šikmosti a výběrové špičatosti.
\end{itemize}
\ul{Exaktní posouzení:}
\begin{itemize}
    \item testy normality (např. Shapirův – Wilkův test, Andersonův-Darlingův test, Lillieforsův test, ...)
\end{itemize}




\subsection*{Jak	ověřit	homoskedasticitu	(shodu	rozptylů)?}
\ul{Empirické posouzení:}
\begin{itemize}
    \item poměr největšího a nejmenšího rozptylu,
    \item vizuální posouzení krabicového grafu.
\end{itemize}
\ul{Exaktní posouzení:}
\begin{itemize}
    \item F – test (parametrický dvouvýběrový test),
    \item Bartlettův test (parametrický vícevýběrový test),
    \item Leveneův test (neparametrický test).
\end{itemize}
\end{document}
